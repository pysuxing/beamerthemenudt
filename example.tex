\documentclass{beamer}

%\usepackage{pgfpages}
%\setbeameroption{show notes on second screen}
%% \setbeameroption{show notes}
\usetheme{nudt}
%\usecolortheme[RGB={02,59,35}]{structure}

\usepackage{tikz}
\usetikzlibrary{matrix, calc}
\usepackage{tikz-3dplot}

% Author, Title, etc.

\title[NUDT Beamer Theme] 
{%
  The NUDT Beamer Theme
}
\subtitle{An Example}

\author[Su Xing]
{
  Su Xing\\
  xingsu@nudt.edu.cn
}

\institute[NUDT]
{
  College of Meteorology and Oceanography\\
  National University of Defence Technology
}

\date[2019]{\today}

% The main document

\begin{document}

\frame{\titlepage}

\begin{frame}
  \frametitle{Outline}
  \tableofcontents
\end{frame}

\section{Overview}

\begin{frame}
  \frametitle{The NUDT Beamer Theme}
  The NUDT beamer theme consists of two components
  \begin{itemize}
  \item The NUDT color theme
  \item The NUDT inner theme
  \end{itemize}
\end{frame}

\section[Color Theme]{The NUDT Color Theme}

\subsection[NUDT VIRS]{The NUDT Visual Image Recognition System}

\begin{frame}
  \frametitle{The NUDT Visual Image Recognition System}
  The NUDT Visual Image Recognition System (VIRS) recommends
  3 color series to be used in publicity materials
  \begin{itemize}
  \item Green\\
    \tikz \fill [nudtcolora01] (0,0) rectangle (10pt, 10pt);
    \tikz \fill [nudtcolora02] (0,0) rectangle (10pt, 10pt);
    \tikz \fill [nudtcolora03] (0,0) rectangle (10pt, 10pt);
  \item Yellow\\
    \tikz \fill [nudtcolorb01] (0,0) rectangle (10pt, 10pt);
    \tikz \fill [nudtcolorb02] (0,0) rectangle (10pt, 10pt);
    \tikz \fill [nudtcolorb03] (0,0) rectangle (10pt, 10pt);
  \item Golden\\
    \tikz \fill [nudtcolorb01] (0,0) rectangle (10pt, 10pt);
    \tikz \fill [nudtcolorb02] (0,0) rectangle (10pt, 10pt);
    \tikz \fill [nudtcolorb03] (0,0) rectangle (10pt, 10pt);
  \end{itemize}
\end{frame}

\begin{frame}
  \frametitle{NUDT VIRS Colors}
  \framesubtitle{Mixed with White}
  \begin{tikzpicture}[x=2.7em,y=1em,
      header/.style={font=\fontsize{8}{8}\selectfont},
      rgb/.style={font=\fontsize{5}{5}\selectfont}]
    \node [header] at (.5,.5) {Color};
    \foreach \name [count=\i] in {A01,A02,A03,B01,B02,B03,C01,C02,C03} {
      \node [header] (\name) at (0.5+\i, 0.5) {\name};
    }
    \node [header] at (.5,-.5) {RGB};
    \foreach \rgb [count=\i] in {195333,214D48,1F341F,E6D832,DEBF12,E9E63B,A18C4D,BDA51E,B8921A} {
      \node [rgb] (\rgb) at (0.5+\i, -0.5) {\#\rgb};
    }
    \foreach \p [count=\j] in {100,90,...,10} {
      \node [header] at (0.5,-.5-\j) {\p\%};
    }
    \foreach \c [count=\i] in {
      nudtcolora01,nudtcolora02,nudtcolora03,
      nudtcolorb01,nudtcolorb02,nudtcolorb03,
      nudtcolorc01,nudtcolorc02,nudtcolorc03} {
      \foreach \p [count=\j] in {100,90,...,10} {
        \fill [\c!\p!white] (\i,-1-\j) rectangle +(.9,1);
      }
    }
  \end{tikzpicture}
\end{frame}

\begin{frame}
  \frametitle{NUDT VIRS Colors}
  \framesubtitle{Mixed with Black}
  \begin{tikzpicture}[x=2.7em,y=1em,
      header/.style={font=\fontsize{8}{8}\selectfont},
      rgb/.style={font=\fontsize{5}{5}\selectfont}]
    \node [header] at (.5,.5) {Color};
    \foreach \name [count=\i] in {A01,A02,A03,B01,B02,B03,C01,C02,C03} {
      \node [header] (\name) at (0.5+\i, 0.5) {\name};
    }
    \node [header] at (.5,-.5) {RGB};
    \foreach \rgb [count=\i] in {195333,214D48,1F341F,E6D832,DEBF12,E9E63B,A18C4D,BDA51E,B8921A} {
      \node [rgb] (\rgb) at (0.5+\i, -0.5) {\#\rgb};
    }
    \foreach \p [count=\j] in {100,90,...,10} {
      \node [header] at (0.5,-.5-\j) {\p\%};
    }
    \foreach \c [count=\i] in {
      nudtcolora01,nudtcolora02,nudtcolora03,
      nudtcolorb01,nudtcolorb02,nudtcolorb03,
      nudtcolorc01,nudtcolorc02,nudtcolorc03} {
      \foreach \p [count=\j] in {100,90,...,10} {
        \fill [\c!\p!black] (\i,-1-\j) rectangle +(.9,1);
      }
    }
  \end{tikzpicture}
\end{frame}

\subsection[Color Usage]{Color Usage in Beamer}

\begin{frame}[fragile]
  \frametitle{Usage of NUDT VIRS Colors}
  \begin{itemize}
  \item The NUDT VIRS colors are assigned proper names via
    the \texttt{\textbackslash{}definecolor} command provided
    by \texttt{xcolor}
    \begin{itemize}
    \item \verb!\definecolor{nudtcolora01}{RGB}{25, 83, 51}!
    \item \verb!\definecolor{nudtcolorb01}{RGB}{230, 216, 50}!
    \end{itemize}
  \item Users can use these colors for typesetting
    \begin{itemize}
      \item \verb!{\color{nudtcolorb01}yellow}!$\rightarrow$ {\color{nudtcolorb01}yellow}
      \item \verb!{\color{nudtcolora01!!\verb!70}light green}!$\rightarrow$ {\color{nudtcolora01!70}light green}
    \end{itemize}
  \end{itemize}
\end{frame}

\section[Inner Theme]{The NUDT Inner Theme}

\begin{frame}
  \frametitle{The Inner Theme}
  \begin{itemize}
  \item Use the inner theme \texttt{rectangules} to make the theme more dominant
  \end{itemize}
\end{frame}

\section[Outer Theme]{The NUDT Outer Theme}

\begin{frame}
  \frametitle{The Outer Theme}
  \begin{itemize}
  \item Use the outter theme \texttt{sidebar} to show sections
    \begin{itemize}
    \item Put the NUDT logoat top of the sidebar
    \item Use \texttt{\color{nudtcolora01}nudtcolora01} as the background color
    \end{itemize}
  \end{itemize}
\end{frame}

\section[Font Theme]{The NUDT Font Theme}

\begin{frame}
  \frametitle{The Font Theme}
  \begin{itemize}
  \item Use the default font theme of \textsc{beamer}
  \end{itemize}
\end{frame}

\end{document}
